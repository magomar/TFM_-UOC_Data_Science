\pagenumbering{roman} 
\setcounter{page}{1} 
\pagestyle{plain}

%%%%%%%%%%%%%%%%
%%% CREDITOS %%%
%%%%%%%%%%%%%%%%
\chapter*{Copyright}

\vspace{1cm}

\begin{figure}[ht]
    \centering
	\includegraphics[scale=1]{ch0/license.png}
\end{figure}

This work is licensed under \href{https://creativecommons.org/licenses/by-nc-sa/3.0/es/deed.en}{Creative Commons Attribution-NonCommercial-ShareAlike 4.0 Spain License (CC BY-NC-SA 4.0 ES)} 

\vspace{1cm}

\begin{figure}[ht]
    \centering
	\includegraphics[scale=1]{ch0/licencia.png}
\end{figure}

Esta obra está sujeta a una licencia de \href{https://creativecommons.org/licenses/by-nc-sa/4.0/es/}{Reconocimiento-NoComercial-CompartirIgual 4.0 España de Creative Commons (CC BY-NC-SA 4.0 ES)}


%%%%%%%%%%%%%
%%% FICHA %%%
%%%%%%%%%%%%%
\chapter*{FICHA DEL TRABAJO FINAL}

\begin{table}[ht]
	\centering{}
	\renewcommand{\arraystretch}{2}
	\begin{tabular}{r | l}
		\hline
		Título del trabajo: & Automated generation of image captions\\
		\hline
        Nombre del autor: & Mario Gómez Martínez\\
		\hline
        Nombre del colaborador/a docente: & Anna Bosch Rué\\
		\hline
        Nombre del PRA: & Jordi Casas Roma\\
		\hline
        Fecha de entrega (mm/aaaa): & 06/2019\\
		\hline
        Titulación o programa: & Máster en Ciencia de Datos\\
		\hline
        Área del Trabajo Final: & Aprendizaje automático\\
		\hline
        Idioma del trabajo: & Inglés\\
		\hline
        Palabras clave & Aprendizaje Profundo, Descripción de Imágenes\\
		\hline
	\end{tabular}
\end{table}

%%%%%%%%%%%%%%%%%%%
%%% DEDICATORIA %%%
%%%%%%%%%%%%%%%%%%%
\chapter*{Dedicatoria}

Dedicado a mi compañera, siempre ahí, para lo bueno y para lo malo, cercana, constante, inspiradora...

%%%%%%%%%%%%%%%%%%%
%%% Agradecimientos %%%
%%%%%%%%%%%%%%%%%%%
% \chapter*{Agradecimientos}

% Quisiera agradecer a...

%%%%%%%%%%%%%%%%
%%% RESUMEN  %%%
%%%%%%%%%%%%%%%%
\chapter*{Abstract}
\addcontentsline{toc}{chapter}{Abstract}

\onehalfspacing

Automatic image captioning, the task of automatically producing a natural-language description for an image, has the potential to assist those with visual impairments by explaining images using text-to-speech systems. However, accurate image captioning is a challenging task that requires integrating and pushing further the latest improvements at the intersection of computer vision and natural language processing fields

This work aims at building an advanced model based on neural networks and deep learning for the automated generation of image captions. 


\vspace{1.5cm}

\textbf{Keywords}: Deep Learning, Artificial Neural Networks, Automated image captioning


\chapter*{Resumen}
\addcontentsline{toc}{chapter}{Resumen}

\onehalfspacing

El subtitulado automático de imágenes, la tarea de producir automáticamente una descripción en lenguaje natural para una imagen, tiene el potencial de ayudar a las personas con discapacidades visuales a explicar las imágenes mediante sistemas de conversión de texto a voz. Sin embargo, el subtitulado preciso de imágenes es una tarea desafiante que requiere integrar y avanzar en la intersección de los campos de procesamiento de lenguaje natural y visión por computador.

Este trabajo pretende desarrollar un modelo basado en redes neuronales y aprendizaje profundo para la generación automática de descripciones de imágenes.


\vspace{1.5cm}

\textbf{Palabras clave}: Aprendizaje Profundo, Redes Neuronales Artificiales, Descripción automática de imágenes