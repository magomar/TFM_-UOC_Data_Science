\chapter{State of the Art}
\label{chapter:state_of_the_art}

Automatic image description can be defined as the task of automatically generating a textual description of an image. It is a challenging problem that encompasses two kind of problems that are also challenging themselves: the problem of understanding an image, and the problem of generating a comprehensive, meaningful and grammatically-correct description of the image. Therefore, it is a task involving research from both Computer Vision (CV) and Natural-Language Processing (NLP).

Both CV and NLP are challenging fields themselves. While both fields share common techniques rooted in artificial intelligence and machine learning, they have historically developed separately, with little interaction between their scientific communities. However, recently there have been an upsurge of interest in problems that require a combination of linguistic and visual information. Besides, the rise of social media in the web has made available a vast amount of multimodal information, like tagged photographs, illustrations in newspaper articles, videos with subtitles, and multimodal feeds on social media.

To exploit the large amounts of multimodal information it is necessary not just to advance in CV and NLP independent research, but to increase the cooperation of both communities and address these tasks as a new kind of tasks

To tackle combined language and vision tasks and to exploit the large amounts of multimodal information, the CV and NLP communities have been increasingly cooperating, for example by organizing combined workshops and conferences.